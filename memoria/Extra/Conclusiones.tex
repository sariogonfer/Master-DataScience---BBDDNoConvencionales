\section*{Conclusiones}
\addcontentsline{toc}{section}{Conclusiones}
\markboth{CONCLUSIONES}{CONCLUSIONES}

MongoDB y Neo4j son dos \gls{BBDD} diferentes en su concepción. Mientras que MongoDB está orientada a documentos, Neo4j está orientada a grafos.

Durante la realización de esta práctica hemos podido sacar algunas conclusiones respecto al uso de ambas \gls{BBDD}:

\begin{itemize}
 \item La carga de los datos en Neo4j ha sido de largo, más complicada que en MongoDB. Esto se debe a las herramientas disponibles para realizar esta carga de datos, al menos, la primera carga, donde el uso de \textit{LOAD CSV} era inviable por el tiempo de carga. En el caso de hacer uso de \textit{neo4j-admin}, es necesario eliminar el grafo en caso de existir, además de ser sensible a errores (en versiones anteriores existía una opción que permitia definir un número máximo de errores permitidos y así poder tratarlos de manera individual a posteriori).
 \item La sintaxis que ofrece \textit{Cypher} resulta más sencilla, en especial a aquellas personas que venga del mundo \gls{SQL}.
 \item Neo4j demuestra su fortaleza a la hora de realizar consultas sobre relaciones entre nodos, pero palidece en aquellas consulta en las que hay que realizar operaciones con los atributos de estos. El uso de índices resulta útil en aquellos atributos que puedan tomar un gran número de posibles valores.
 \item MongoDB dispone de un framework para hacer agregaciones muy potente, aunque su curva de aprendizaje es muy empinada al principio.
 \item En la actualidad, MongoDB está presentando actualizaciones constantemente, ampliando a través de nuevas funcionalidades que los usuarios que venían del mundo \gls{SQL} parecían echar de menos. Esto está consiguiendo que la expansión de utilización de MongoDB sea a un ritmo frenético.
\end{itemize}