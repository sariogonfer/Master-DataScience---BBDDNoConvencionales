\section{MongoDB}

La primera parte de la práctica estará enfocada en la \gls{BBDD} no ralacional MongoDB. Esta se trate de la \gls{BBDD} más importante actualmente. En el momento de escribir este texto, se encuentra en el 5to de \gls{BBDD} más utilizadas según la web \url{https://db-engines.com/en/ranking} y solo por detrás de las \gls{BBDD} relacionales.

MongoDB tiene las siguientes características:

\begin{itemize}
 \item Es schemas less, es decir, no es preciso definir un esquema de datos. Aunque no por ello, tenemos que dejar de saber como son almacenados los datos.
 \item Usa el formato BSON para almacenar los datos, haciendo que sea facil su uso desde lenguajes de programación como Python y JavaScript.
 \item Permite realizar agregaciones. Incluye el framework aggregation el cuál le permite hacer operaciones realmente potentes.
 \item Podemos crear índices (incluyendo índices parciales) de cualquiera de sus campos, lo que acelera notablemente las busquedas.
 \item Su característica de \textit{sharding} le permite ser muy escalable.
 \item Cuenta con una gran comunidad de desarrolladores detrás lo cual es una garantía de futuro.
 \item y muchas más.
\end{itemize}

Para la realización de esta parte de la práctica vamos a partir de tres documentos \gls{JSON}, los cuales podemos ver la forma de obtenerlos en el anexo \ref{sec:parser}.

Cada uno de estos documentos contiene la información sobre los tres tipos de publicaciones que vamos a usar:

\begin{itemize}
 \item Articles.
 \item Inproceedings.
 \item Incollections.
\end{itemize}

En nuestro caso, hemos hecho uso de una base de datos MongoDB encapsulada en un contenedor docker. Es posible realizarlo mediante una instalación nativa sobre linux pero hemos preferido la otra opción. Si se desea utilizar el mismo contenedor que hemos utilizado en nuestro caso, basta con ejecutar el siguiente comando:

\begin{minted}[
frame=single]{js}
sudo docker run --name practica-bbdd-mongo -d -p 27017:27017 mongo
\end{minted}

Una vez lanzado el comando indicado, para volver a arrancar el contenedor basta con ejecutar el comando start con el nombre del contenedor creado:

\begin{minted}[
frame=single]{js}
sudo docker start practica-bbdd-mongo
\end{minted}
